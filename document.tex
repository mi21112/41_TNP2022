\documentclass[12pt]{article}
\usepackage[utf8]{inputenc}
\usepackage{graphicx}
\usepackage[serbian]{babel}

\title{Korisni alati na androidu za učenje programiranja}
\author{Danilo Nikolaš, Luka Nedeljković, Nemanja Kelečević, Ivan Vlahović}
\date{15. novembar 2022.}

\begin{document}

\maketitle
\pagebreak

\section{Uvod}
Poslednjih godina prisutan je sve veći broj aplikacija i okruženja za android, koje pružaju razne mogućnosti.
Pomenute aplikacije mogu pomoći, kako početnicima da savladaju prve korake u programiranju, tako i iskusnijim programerima da usavrše svoje veštine na efikasan i zabavan nacin.
U narednom tekstu predstavićemo aplikacije koje su nam se učinile kao najbolji izbor, medju njima su Android Studio,Unity 3D... \

\section{Android Studio}
Android Studio je jedan od najzastupljenijih i najbitnijih alata za Android programere. To je \textbf{integrisano razvojno okruženje} (IDE) za Android, zasnovano na softveru JetBrains IntelliJ IDEA.  \\
Službeni programski jezik je \textbf{Java}, ali su takodje podržani C++, a i od nedavno Kotlin. Razvojno okruženje poseduje kompajler koji omogućava stvaranje APK datoteka, XML uređivač, kao i funkciju za "prikaz dizajna" pomoću koje vizuelnim putem omogućava organizovanje elemenata na ekranu. \\
Android Studio nudi kompletan set dodatnih alata koji programeri mogu efikasno da iskoriste, kao što su AVD Manager, Android Debug Bridge, monitor za praćenje performansi itd.\\
Neke od prednosti Android Studia:
\begin{itemize}
\item{Veoma brzo i efikasno pokretanje aplikacije}
\item{Inteligentno uredjivanje koda} 
\item{Emulator sa mnogo mogućnosti}
\item{Može se koristiti za sve android uređaje}
\end{itemize}

\begin{figure}[ht!]
    \centering
    \includegraphics[scale=0.2]{android_studio_interface.png}
    \caption{Razvojno okruženje Android Studia}
\end{figure}


\subsection{Neki dodatni alati Android Studia}
Alat AVD Manager (Android virtualni uređaj) je uključen u Android Studio i u osnovi je emulator koji omogućava pokretanje Android aplikacija na računaru. Zbog toga je vrlo koristan alat jer omogućava brzo testiranje aplikacija bez potrebe za instaliranjem na fizičke uređaje. Pored toga, omogućava simulaciju različitih veličina ekrana, specifikacija, verzija Androida ... Sve ovo i još više omogućava vam optimizaciju aplikacije za njeno izvršavanje na bilo kojem uređaju.

\begin{figure}[ht!]
    \centering
    \includegraphics[scale=0.2]{AVD_manager.png}
    \caption{AVD manager}
\end{figure}

Android Debug Bridge (ADB) je još jedan od korisnih alata Android Studia. U suštini, ovaj alat omogućava terminalni interfejs za interakciju sa telefonom. Kako je Android platforma bazirana na Linuxu, terminalni pristup je jedini način dobijanja admin pristupa. ADB omogućuje most između uređaja i kompjutera.

\begin{figure}[ht!]
    \centering
    \includegraphics[scale=0.1]{adb.png}
    \caption{Android Debug Bridge}
\end{figure}


\section{Unity 3D}

Unity 3D je viseplatformsko okruzenje za razvoj igara sa ugradjenim IDE (Integrisano razvojno okruzenje) – softverska aplikacija koja pruza pogodnosti programerima pri razvoju softvera. IDE se sastoji od uredjivaca koda, ispravljaca gresaka I takodje poseduje algoritme automatizacije koda sto 
znatno olaksava rad programerima.

Iako je prva verzija unity-a, predstavljena na Apple konferenciji “Worldwide Developers Conference” 2005. godine, prvobitno bila namenjena da funkcionise na Mac racunarima, vremenom se razvila I postula dostupna za sve platforme. 

\begin{table}[ht!]
\begin{tabular}{|l|l|}
\hline
\multicolumn{1}{|c|}{\textbf{Unity 2.0 (2007)}} & Olaksan rad vise programera na istom projektu.\\ \hline
\textbf{Unity 3.0 (2010)}                       & Poboljsane graficke perfomanse za Desktop racunare.                               \\ \hline
\textbf{Unity 4.0 (2012)}                       & Podrska za Adobe Flash, saradnja sa Facebook-om. \\ \hline
\textbf{Unity 2017}                             & Poboljsan rad sa animacijama i kamerama. \\ \hline
\end{tabular}
\end{table}

Unity razvojno okruzenje koristi jezik C# kao primarni programski jezik.
Kako je poznavanje jezika C# I njegovih funkcionalnosti osnova programiranja I kako vecina programera jako dobro poznaje ovaj jezik, sa lakocom se moze naviknuti na rad u Unity-ju. Takodje, jezik C# ima veoma veliku zajednicu pa se moze ocekivati da je problem sa kojim se korisnik srece vec obradjen negde u okviru ove zajednice. Samim tim programer ima veliku podrsku I literaturu koja mu je dostupna pri razvoju softvera u ovom okruzenju. Ucenje programiranja se zasniva na poznavanju koncepata ovog jezika, svaki pocetnik ce vrlo lako savladati principe Unity okruzenja. 

Unity 3D ima siroku primenu na velikom broju platformi.
Koju god platformu preferirali, velike su sanse da je Unity podrzava.

\begin{figure}[ht!]
    \centering
    \includegraphics[scale=0.2]{platforme.png}
    \caption{Dostupne platforme}
\end{figure}


Unity podrzava I dvodimenzionalno (2D) i trodimenzionalno(3D) programiranje. Vecina programa u ovakvim situacijama favorizuje jednu opciju I naklonjena je vise ka jednoj od njih ali to nije slucaj sa Unity razvojnim okruzenjem. Postoji mnostvo alata, skracenica I opcija koje omogucavaju olaksan rad pocetnicima ali I onima sa vise iskustva. 

Interfejs Unity okruzenja je jednostavan I intuitivan. 
Ovaj interfejs svakako spada u grupu onih koji su najlaksi za koriscenje, usavrsavanje I ucenje. Preglednost I jednostavnost jako su bitne karakteristike svakom programeru pocetniku. 

Unity u okviru svog okruzenja sadrzi I \textbf{Asset Store} – platforma na kojoj programeri mogu deliti svoje kodove I biblioteke. Mnoge funkcionalnosti su dostupne potpuno besplatno, takodje programeri svoje kodove mogu I prodavati u okviru ove platforme. Pocetnici mogu koristiti ovu prodavnicu I bilbioteke ugradjene u njoj da razviju svoj softver od nule sto omogucava lakse upoznavanje sa konceptima programiranja u Unity okruzenju.

 Unity ima jos jednu pogodnost za pocetnike, pruza potpuno besplatne obuke 
na svom oficijalnom sajtu. Sve neophodne dokumentacije I interfejsi su takodje besplatni I dostupni na internetu.

\section{Unreal Engine}

\section{Basic4Android}


\section{Zaključak}

\pagebreak
\begin{thebibliography}{}
\end{thebibliography}

\end{document}
